\documentclass{article}
\usepackage[utf8]{inputenc}
\usepackage{verbatim}

\title{Introduction to\LaTeX }
\author{Medet Can Akuş}
\date{\today}

\begin{document}
\maketitle
\section{Things You Need to Know}
Tex is a computer program created by Donald E. Knuth. Main purpose of the TeX is to typeset and write\
mathematical formulas with ease. \\
\LaTeX\ built on the TeX language it further improves typesetting abilities by introducing predefined layout and macros.
\LaTeX\ originally created by Leslie Lamport.\\
\LaTeX\ assumes the role of the book designer.
\subsection{\LaTeX\ Input Files}
\begin{itemize}
    \item \LaTeX\ files are simply text files containing instructions to how to structure to the document
    \item A \LaTeX\ file ends with a .tex 
\end{itemize}
\subsubsection{Spaces}
\begin{itemize}
    \item Whitespace characters such as blank spaces, tabs are treated uniformly as "space" by \LaTeX\
    \item Several consecutive white space characters are treated as one "space".
    \item Beginning of the line with a white space is ignored and single new line simply treated as a white space
    \item To instruct \LaTeX to create a new line use "\verb|\\|"
    \item An Empty line between two lines of text defines the end of a paragraph.
    \item More than one empty line treated as a line break.
\end{itemize}
\newpage
\subsubsection{Special Characters}
\begin{itemize}
    \item Followings are the special characters in \LaTeX\
    \begin{itemize}
        \item \verb|#, $, ^, &, _, {, }, ~, \|
    \end{itemize}
    \item These special characters have special meaning to \LaTeX\
    \item These can be escaped by prefixing them with \verb|\|
    \item To use backslash as literally use \verb|\textbackslash|
\end{itemize}
\subsubsection{\LaTeX\ Commands}
\begin{itemize}
   \item \LaTeX\ commands are case sensitive. They start with a backslash.
   \item \LaTeX\ commands are terminated by a space, a number or any other "non-letter"
   \item Many \LaTeX\ commands have starred variant.
   \item \LaTeX\ ignores whitespace after commands. To force a white space you can use\
   empty parenthesis "\{\}" 
   \item Some commands require a parameters, which has to be given between curly braces.
   \item Some commands take optional parameters, which are inserted after the command name in square brackets
   \item A \LaTeX\ command usually take take the form; \\ \verb|\command[optional parameters]{parameters}|
\end{itemize}
\subsubsection{Comments}
\begin{itemize}
    \item \verb|%| symbol signifies a comment
    \item \LaTeX{} compiler ignores anything when it encounters \verb|%| symbol until the end of the line.
    \item for multiline comments use \verb|\comment| environment by importing verbatim package.
\end{itemize}
\subsubsection{File Structure}
\begin{itemize}
    \item \LaTeX\ expects a certain structure to process the file.
    \item Every file should start with a \verb|\documentclass{}| command!
    \item To load a package use \verb|\usepackage{}| packages provide new features to your document.
    \item To start writing the document use \verb|\begin{document}| and to end it use \verb|\end{document}|
    \item Anything comes before \verb|\begin{document}| and after the \verb|\documentclass{}| is the preamble.
\end{itemize}
\subsubsection{Layout of the Document}
\begin{itemize}
    \item Type of document is specified with \verb|\documentclass[options]{class}| command.
    \item List of classes;
    \begin{itemize} 
        \item article: for articles in scientific journals
        \item proc: a class for proceedings.
        \item report: for longer reports containing several chapters, small books.
        \item books: for real books.
        \item slides: for slides.
    \end{itemize}
    \item List of optional parameters;
    \begin{itemize}
        \item 10pt, 11pt, 12pt sets the font size.
        \item a4paper, letterpaper defines the paper size.
        \item fleqn Typesets displayed formulae left-aligned instead of centered.
        \item onecolumn, twocolumn specifies column format.
        \item twoside, oneside specifies whether double or single sided output should be generated.
    \end{itemize}
\end{itemize}
\subsection{Packages}
\begin{itemize}
    \item If you want to extent your document with graphics, mathematical formulas you will probably need to use packages.
    \item Packages are activated with \verb|\usepackage[options]{package}|
    \item Package command are written inside of the preamble section of your source code. 
    \item If you use any of the linux distributions you can invoke \verb|$ texdoc packagename| command to see documentation of the package
\end{itemize}
\subsection{Page Styles}
\begin{itemize}
    \item \LaTeX\ supports three predefined header/footer combinations --so called page styles.
    \item \verb|\pagestyle{style}| command tells which page style to use.
    \item If you intent to be more specific you can use \verb|\thispagestyle{style}| it only applies to page corresponding to the command.
\end{itemize}
\subsection{Files You May Encounter}
\begin{itemize}
   \item Some packages are so essential they are distributed with \LaTeX\
   \item Here is a brief description of each one of them;
   \begin{itemize}
      \item doc: Allows to documentation of \LaTeX\ programs.
      \item exscale: Provides a scaled versions of the math extension font.
      \item fontenc: Specifies which font encoding \LaTeX\ should use.
      \item ifthen: Provides commands of the form "if...then do...otherwise do...."
      \item latexsym: To access the \LaTeX\ symbol font, you should use latexsym package.
      \item makeidx: Provides commands for producing indexes.
      \item syntonly: Processes a document without a typesetting it.
      \item inputenc: Allows the specification of input encoding such as utf8.
   \end{itemize}
   \item Here is a brief description of each page styles;
   \begin{itemize}
      \item plain: prints the page numbers on bottom of the page, in middle of the footer. This is the default page style. 
      \item headings: prints the current chapter heading and the page number in the header on each page, while footer remains empty.
      \item empty: sets both the header and the footer to be empty.
   \end{itemize}
   \item When you work with \LaTeX\ you will encounter handful of intermediate files with different extension. \\
   Here is a brief description of each one of them;
   \begin{itemize}
    \item .tex: \LaTeX\ and \TeX\ input file. Can be compiled with latex command. 
    \item .sty: \LaTeX\ macro package. Load this into your \LaTeX\ document using \verb|\usepackage| command
    \item .dtx: \LaTeX\ Documented \TeX. This is the main distribution format for \LaTeX\ style files.
    \item .ins: The installer for the files contained in the matching .dtx file
    \item .cls: Class files define what your document looks like.
    \item .fd: Font description file telling \LaTeX\ about new fonts.
   \end{itemize}
   \item Following files are generated when you run \LaTeX\ on your input file.
   \begin{itemize}
       \item .dvi: Device independent file. This is the main result of a classical \LaTeX\ compile run.
       \item .log: Gives a detailed account of what happened during the last compilation.
       \item .toc: Stores all your section headers. It gets read in for the next compiler run and is used to produce the table of contents.
       \item .lof: like the \verb|.toc| file but for the figures.
       \item .lot: like the \verb|.toc| file but for the tables.
       \item .aux: Another file that transports information from one compiler run to the next. Aux files are used for store information associated with cross references.
       \item .idx: Stores the indexing information.
       \item .ind: The processed index file. Ready for inclusion into your document on the next compile cycle.
       \item .ilg: Log file for index generation process.
   \end{itemize}
\end{itemize}
\subsection{Big Projects}
\begin{itemize}
   \item You can load your other input documents into the current context using \verb|\include{filename}| 
   \item \verb|\includeonly{filename, filename, ...}| command can be used in the preamble. It allows you to instruct \LaTeX\ to only input some of the included files.
   \item \verb|include{filename}| command includes the given file on a new page. Sometimes this is not desirable for these type of situations you can use \verb|\input{filename}|
\end{itemize}
\section{Typesetting Text}
\begin{itemize}
   \item d 
\end{itemize}
\end{document}